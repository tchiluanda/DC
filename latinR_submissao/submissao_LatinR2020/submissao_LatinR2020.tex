\documentclass[letterpaper]{report}
%\usepackage[utf8]{inputenc}
\usepackage[T1]{fontenc}
\usepackage{RJournal}
\usepackage{amsmath,amssymb,array}
\usepackage{booktabs}

%% load any required packages here

\usepackage[portuguese]{babel}
\usepackage{graphicx}

\hypersetup{pdftitle={Alavancando o poder do RMarkdown com as linguagens da Web e D3.js para
produzir histórias de dados envolventes sobre Finanças Públicas},
            pdfkeywords={R Markdown; Data Visualization; Data Storytelling; Web Standards; r2d3; Public Finance; Government}}


\hypersetup{pdfauthor={Tiago Pereira, Fernando Barbalho, Jordao Goncalves, Lucas Leite}}


%\usepackage[hidelinks]{hyperref}

\urlstyle{same}  % don't use monospace font for urls
\usepackage{color}
\usepackage{fancyvrb}
\newcommand{\VerbBar}{|}
\newcommand{\VERB}{\Verb[commandchars=\\\{\}]}
\DefineVerbatimEnvironment{Highlighting}{Verbatim}{commandchars=\\\{\}} 
% Add ',fontsize=\small' for more characters per line
\usepackage{framed}
\definecolor{shadecolor}{RGB}{248,248,248}
\newenvironment{Shaded}{\begin{snugshade}}{\end{snugshade}}
\newcommand{\AlertTok}[1]{\textcolor[rgb]{0.94,0.16,0.16}{#1}}
\newcommand{\AnnotationTok}[1]{\textcolor[rgb]{0.56,0.35,0.01}{\textbf{\textit{#1}}}}
\newcommand{\AttributeTok}[1]{\textcolor[rgb]{0.77,0.63,0.00}{#1}}
\newcommand{\BaseNTok}[1]{\textcolor[rgb]{0.00,0.00,0.81}{#1}}
\newcommand{\BuiltInTok}[1]{#1}
\newcommand{\CharTok}[1]{\textcolor[rgb]{0.31,0.60,0.02}{#1}}
\newcommand{\CommentTok}[1]{\textcolor[rgb]{0.56,0.35,0.01}{\textit{#1}}}
\newcommand{\CommentVarTok}[1]{\textcolor[rgb]{0.56,0.35,0.01}{\textbf{\textit{#1}}}}
\newcommand{\ConstantTok}[1]{\textcolor[rgb]{0.00,0.00,0.00}{#1}}
\newcommand{\ControlFlowTok}[1]{\textcolor[rgb]{0.13,0.29,0.53}{\textbf{#1}}}
\newcommand{\DataTypeTok}[1]{\textcolor[rgb]{0.13,0.29,0.53}{#1}}
\newcommand{\DecValTok}[1]{\textcolor[rgb]{0.00,0.00,0.81}{#1}}
\newcommand{\DocumentationTok}[1]{\textcolor[rgb]{0.56,0.35,0.01}{\textbf{\textit{#1}}}}
\newcommand{\ErrorTok}[1]{\textcolor[rgb]{0.64,0.00,0.00}{\textbf{#1}}}
\newcommand{\ExtensionTok}[1]{#1}
\newcommand{\FloatTok}[1]{\textcolor[rgb]{0.00,0.00,0.81}{#1}}
\newcommand{\FunctionTok}[1]{\textcolor[rgb]{0.00,0.00,0.00}{#1}}
\newcommand{\ImportTok}[1]{#1}
\newcommand{\InformationTok}[1]{\textcolor[rgb]{0.56,0.35,0.01}{\textbf{\textit{#1}}}}
\newcommand{\KeywordTok}[1]{\textcolor[rgb]{0.13,0.29,0.53}{\textbf{#1}}}
\newcommand{\NormalTok}[1]{#1}
\newcommand{\OperatorTok}[1]{\textcolor[rgb]{0.81,0.36,0.00}{\textbf{#1}}}
\newcommand{\OtherTok}[1]{\textcolor[rgb]{0.56,0.35,0.01}{#1}}
\newcommand{\PreprocessorTok}[1]{\textcolor[rgb]{0.56,0.35,0.01}{\textit{#1}}}
\newcommand{\RegionMarkerTok}[1]{#1}
\newcommand{\SpecialCharTok}[1]{\textcolor[rgb]{0.00,0.00,0.00}{#1}}
\newcommand{\SpecialStringTok}[1]{\textcolor[rgb]{0.31,0.60,0.02}{#1}}
\newcommand{\StringTok}[1]{\textcolor[rgb]{0.31,0.60,0.02}{#1}}
\newcommand{\VariableTok}[1]{\textcolor[rgb]{0.00,0.00,0.00}{#1}}
\newcommand{\VerbatimStringTok}[1]{\textcolor[rgb]{0.31,0.60,0.02}{#1}}
\newcommand{\WarningTok}[1]{\textcolor[rgb]{0.56,0.35,0.01}{\textbf{\textit{#1}}}}

\providecommand{\keywords}[1]{\noindent\textbf{Palabras clave:} #1}
\providecommand{\tightlist}{%
\setlength{\itemsep}{0pt}\setlength{\parskip}{0pt}}


\begin{document}

%% do not edit, for illustration only
\sectionhead{Alavancando o poder do RMarkdown com as linguagens da Web e D3.js para
produzir histórias de dados envolventes sobre Finanças Públicas}
\year{2020}

\begin{article}

\title{Alavancando o poder do RMarkdown com as linguagens da Web e D3.js para
produzir histórias de dados envolventes sobre Finanças Públicas}

\author{
Tiago Pereira , 
Fernando Barbalho , 
Jordao Goncalves , 
Lucas Leite }


\maketitle


\keywords{ R Markdown  -  Data Visualization  -  Data Storytelling  -  Web Standards  -  r2d3  -  Public Finance  -  Government }

\hypertarget{introduuxe7uxe3o}{%
\section{Introdução}\label{introduuxe7uxe3o}}

O R Markdown permite integrar a análise e a comunicação de dados,
costurando texto e códigos numa míriade de formatos de saída, desde
artigos e slides até livros e aplicações web interativa.

A simplicidade da linguagem markdown, o poder do R e a versatilidade de
formatos de saída fazem com que usuários de R possam produzir, num mesmo
ambiente, um fluxo completo e reproduzível, desde a importação de dados
brutos até a construção de um documento visualmente atraente.

A figura abaixo, desenhada por Allison Horst, é uma excelente ilustração
desse processo:

\includegraphics{"RMarkdown.jpg"}

Em particular, o formato de saída HTML permite que usuários sem
conhecimentos de web design produzam páginas web completas, com design
agradável e prontas para publicação e comunicação do resultado de suas
análises.

\hypertarget{indo-aluxe9m}{%
\section{Indo além}\label{indo-aluxe9m}}

No entanto, é possível criar histórias de dados ainda mais envolventes e
atraentes mergulhando um pouco mais fundo nas linguagens e tecnologias
envolvidas na geração de documentos HTML para a Web a partir do R
Markdown.

Uma das características do R Markdown que lhe conferem tamanha
versatilidade é a possibilidade de inclusão de ``chunks'' de códigos
escritos em outras linguagens além de R.

Por exemplo, para personalizar um pouco mais a aparência da página web,
é fundamental usar \emph{CSS} (``Cascading Style Sheets''), a linguagem
usada para descrever a apresentação de páginas Web, incluindo cores,
layouts e fontes. Códigos em CSS podem ser incluídos no próprio arquivo
R Markdown. No documento HTML gerado, esses chunks serão incluídos como
blocos \texttt{\textless{}style\textgreater{}}.

Para incluir novos elementos visuais que possam ser formatados pelo
código em CSS (boxes explicativos em meio ao texto, como no caso do
nosso projeto), é possível utilizar a própria sintaxe do Pandoc, em
muitos casos, como no caso de ``fenced divs'' e ``bracketed
spans''(\emph{Pandoc User Guide}, n.d.). Para os demais casos, existe a
possibilidade de se incluir o código HTML puro no corpo do texto.

Além disso, é possível acrescentar interatividade à página com o uso de
Javascript, cujos códigos também podem ser incluídos como chunks. Por
fim, graças ao pacote \texttt{r2d3}, pode-se ainda incluir chunks de
D3.js(\emph{Data-Driven Documents}, n.d.), a principal biblioteca em
Javascript para construção de visualizações interativas e sem formatos
pré-determinados.

\hypertarget{o-projeto-comunicando-finanuxe7as-puxfablicas-de-uma-forma-mais-amiguxe1vel-para-a-sociedade}{%
\section{O Projeto: comunicando Finanças Públicas de uma forma mais
amigável para a
sociedade}\label{o-projeto-comunicando-finanuxe7as-puxfablicas-de-uma-forma-mais-amiguxe1vel-para-a-sociedade}}

Neste projeto, utilizamos as demonstrações financeiras declaradas ao
Tesouro Nacional pelo governo federal, pelos governos estaduais e por
93\% dos 5.570 municípios brasileiros. Com base nesses dados,
construímos com R Markdown uma página Web apresentando as informações
das dívidas públicas dos governos brasileiros numa estrutura narrativa e
envolvente. Para isso utilizamos diversos recursos.

Toda a importação, preparação, processamento, agrupamento e tratamento
dos dados foi feita com R, utilizando extensivamente os pacotes do
\texttt{tidyverse}.

Para encontrar um equilíbrio entre simplicidade e rigor conceitual no
texto, deixamos o texto principal mais simples, mas incluímos a
possibilidade de o usuário clicar em termos mais técnicos para que a
página exiba quadros com explicações adicionais.

\includegraphics{"boxes.png"}

Isso é possível com uma combinação do uso da sintaxe do Pandoc para
criar elementos customizados, chunks de CSS (para a formatação dos
elementos) e Javascript (para acrescentar interatividade).

\includegraphics{"chunks.png"}

Para os gráficos estáticos, utilizamos \texttt{ggplot2} (exemplos
abaixo).

\includegraphics{"plots.png"}

Além disso, como mencionado e indicado acima, como experimento fizemos
uso do pacote \texttt{r2d3} para construir gráficos em D3 diretamente no
R Markdown. Uma vantagem de usar essa abordagem é permitir a utilização
de gráficos interativos no próprio formato HTML, sem a necessidade de um
servidor Shiny.

Finalmente, utilizamos o chunk de CSS para definir o layout geral da
página, fontes, cores, espaçamentos, margens etc. Gráficos bonitos
feitos com o \texttt{ggplot2} auxiliam bastante na tentativa de tornar o
conteúdo mais interessante. Mas investir um pouco de tempo para conhecer
as possibilidade de CSS, principalmente, pode trazer excelentes retornos
para a qualidade final do produto.

\hypertarget{reflexuxf5es}{%
\section{Reflexões}\label{reflexuxf5es}}

Assim, para poder combinar ``Finanças Públicas'' com ``histórias de
dados envolventes'', ou seja, para tornar esse assunto menos árido e
mais interessante, decidimos fazer uso do potencial das linguagens dos
padrões da Web, reunidas no ambiente ao qual já estamos tão bem
familiarizados para realizar nossas análises de dados.

Dessa forma, é possível, num único documento de texto simples
\texttt{.Rmd}, escrever textos, importar e manipular dados com R, gerar
visualizações complexas e atraentes com \texttt{ggplot2}, construir
elementos visuais com markdown e HTML, formatar a aparência desses
elementos com CSS, incluir interatividade com Javascript, embutir
visualizações interativas e completamente customizadas com D3.

Além da versatilidade, a vantagem do R Markdown é a de integrar todo o
fluxo desde a importação dos dados originais até a criação da página Web
final. Assim, nas atualizações futuras dos dados, basta ``costurar''
novamente o arquivo \texttt{.Rmd} para obtermos a página Web atualizada.

A apresentação ``Of Teacups, Giraffes, \& R Markdown'', de Desirée de
Leon(\emph{Of Teacups, Giraffes, \& R Markdown} 2020), na RStudio::conf
2020, serviu de grande inspiração para este trabalho. Como Desirée diz,
na apresentação: ``a preocupação com o design não apenas causa uma
melhor impressão inicial, mas pode melhorar a legibilidade, a navegação
e a experiência geral do usuário à medida em que ele consome o
conteúdo''.

O código está disponível em: https://github.com/tchiluanda/DC

A aplicação está on-line em: https://tchiluanda.github.io/DC/

\hypertarget{refs}{}
\leavevmode\hypertarget{ref-D3}{}%
\emph{Data-Driven Documents}. n.d. \url{https://d3js.org/}.

\leavevmode\hypertarget{ref-Desiree}{}%
\emph{Of Teacups, Giraffes, \& R Markdown}. 2020.
\url{https://rstudio.com/resources/rstudioconf-2020/of-teacups-giraffes-r-markdown/}.

\leavevmode\hypertarget{ref-Pandoc}{}%
\emph{Pandoc User Guide}. n.d.
\url{https://www.pandoc.org/MANUAL.html\#divs-and-spans}.

\address{Tiago Pereira\\
National Treasury of Brazil\\
\email{tiagombp@gmail.com}}

\address{Fernando Barbalho\\
National Treasury of Brazil\\
\email{fernando.barbalho@tesouro.gov.br}}

\address{Jordao Goncalves\\
National Treasury of Brazil\\
\email{jordao.m.goncalves@tesouro.gov.br}}

\address{Lucas Leite\\
National Treasury of Brazil\\
\email{lucas.leite@tesouro.gov.br}}



\end{article}
\end{document}

